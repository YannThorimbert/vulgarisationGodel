recursivement axiomatisable:
je prends TOUTES les propositions vraies de l’arithmétique, et je crée un système d’axiomes dans lequel chacune de ces propositions est un axiome. J’obtiens alors un système d’axiomes (cohérent?) pas forcément très intéressant, mais qui semble violer le théorème de Gödel, puisque toute proposition vraie de l’arithmétique y est démontrable

Ce système d’axiomes porte un nom, on l’appelle “True Arithmetics”, mais outre le fait qu’il est totalement inutile (dans le sens redondant?), il a un problème : si je vous donne une proposition quelconque, vous n’avez pas de moyen simple(POURQUOI?) de dire si oui ou non cette proposition fait partie des axiomes du système. De fait un tel système d’axiome est totalement inutilisable !

C’est pour cela qu’en pratique, on s’impose de travailler avec des systèmes d’axiomes récursifs, c’est-à-dire faits d’axiomes qui sont potentiellement en nombre infini (POURQUOI?), mais tels que pour toute proposition, on puisse dire en un temps fini si oui ou non la proposition fait partie des axiomes.

Le théorème de Gödel porte justement sur les systèmes d’axiomes récursifs


on peut prouver l'indécidabilité d'une proposition : e.g algo de l'arret